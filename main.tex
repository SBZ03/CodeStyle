\documentclass{article}

\usepackage[utf8]{inputenc}
\usepackage[russian]{babel}
\usepackage{authblk}
\usepackage{cite}
\usepackage[usenames]{color}

\title{\textbf{CodeStyle}}
\author{ИА-031 Шолохов Владислав}
\affil{github: @SBZ03; email: sholokhov.02@list.ru}
\date{February 2022}

\begin{document}

\maketitle

\section{Введение}
Code style как стандарт разработки. Code style помогает обеспечивать линейное развитие проекта, значительно ускоряет адаптацию новых сотрудников и, в целом, формирует и воспитывает культуру разработки.\cite{1}

\section{\underline{Язык программирования C/C++}}

\subsection{Отступы и пробелы}
Для отступа используется табуляцию (размер таба 4 пробела).\\
\\\\
\begin{lstlisting}
\begin{tabular}{ | l | }
\hline
\\
1 \textcolor{blue}{int} main()\{\\
2 \qquad ...\\
3 \qquad \textcolor{blue}{return} 0;\\
4 \}\\
\\
\hline
\end{tabular}\\
\\\\
Операторы и операнды разделяются пробелом.\\
\\\\
\begin{tabular}{ | l | }
\hline
\\
1 \textcolor{blue}{int} a = (b + c) * e;\\
\\
\hline
\end{tabular}\\
\\\\
Перед открывающей скобкой и после знаков препинания используется пробел.\\
\\
\begin{tabular}{ | l | }
\hline
\\
1 \textcolor{blue}{for} (...; ...; ...)\\
\\
\hline
\end{tabular}\\
\\\\
Перед двоеточием в конце строки(если оно нужно) пробел не ставится.
\end{lstlisting}

\subsection{Объявление переменных}
Переменным даются описательные имена, такие как firstName или counter.\\
\\
Избегаются однобуквенные названия вроде x или c, за исключением итераторов вроде i,j,k.\\
\\
Если переменная используется лишь внутри определенного if, то делайте её локальной, объявляя в том же блоке кода, а не глобальной.\\
\\
Выбирается подходящий тип данных для ваших переменных. Если переменная содержит лишь целые числа, то она определяется как int, а не double.\\
\\\\
\begin{lstlisting}
\begin{tabular}{ | l | }
\hline
\\
1 \textcolor{blue}{int} counter;\\
2 \textcolor{blue}{double} average;\\
\\
\hline
\end{tabular}\\
\\\\
Используется текстовая строка, стандартная для C++.\\
\\\\
\begin{tabular}{ | l | }
\hline
\\
1 string str = "Hello there;\\
\\
\hline
\end{tabular}\\
\\\\
Если определенная константа часто используется в коде, то она обохначается как const и мы всегда ссылаемся на данную константу, а не на её значение.\\
\\\\
\begin{tabular}{ | l | }
\hline
\\
1 \textcolor{blue}{const int} VOTINGAGE = 18;\\
\\
\hline
\end{tabular}\\
\\\\
\end{lstlisting}

\subsection{Функции}
Между функциями и группами выражений остаётся пустая строка.\\
\\\\
\begin{lstlisting}
\begin{tabular}{ | l | }
\hline
\\
1 \textcolor{blue}{void} fun1(...)\{\\
2 \qquad ...\\
3 \}\\
4 \qquad\qquad\qquad//пустая строка\\
5 \textcolor{blue}{void} fun2(...)\{\\
6 \qquad ...\\
7 \}\\
\\
\hline
\end{tabular}\\
\\\\
Имена функций должны быть логическими и записаны с нижнего регистра, но могут быть записаны в смешанном регистре.\\
\\
Порядок их написания не должен вызывать ошибок, функции должны быть записаны в логическом порядке\\
\\
\begin{tabular}{ | l | }
\hline
\\
1  \textcolor{blue}{int} createMatrix(...)\{\\
2  \qquad ...\\
3  \}\\
4 \\
5  \textcolor{blue}{void} \textcolor{yellow}{deleteMatrix}(...)\{\\
6  \qquad ...\\
7  \}\\
8 \\
9  \textcolor{blue}{void} printMatrix(...)\{\\
10 \qquad ...\\
11 \qquad \textcolor{yellow}{deleteMatrix}(...);\\
12 \}\\
\\
\hline
\end{tabular}\\
\end{lstlisting}

\subsection{Операторы управления}
\begin{lstlisting}
Итерационные операторы(while, for, do/while).\\
Используются отступы и переходы на новую строку, если есть в этом необходимость. Для счётчика используются итераторы вроде i,j,k.\\
\\\\
\begin{tabular}{ | l | }
\hline
\\
1 \textcolor{blue}{while} (i < N)\{\\
2 \qquad ...\\
3 \}\\
\\
\hline
\end{tabular}\\
\\\\
\begin{tabular}{ | l | }
\hline
\\
1 \textcolor{blue}{for} (int j = 0; j < N; j++)\{\\
2 \qquad ...\\
3 \}\\
\\
\hline
\end{tabular}\\
\\\\
\begin{tabular}{ | l | }
\hline
\\
1 \textcolor{blue}{do}\{\\
2 \qquad ...\\
3 \} while(k < N);\\
\\
\hline
\end{tabular}\\
\\\\
Операторы выбора(if else,switch).
Используются отступы и переходы на новую строку, если есть в этом необходимость. Для if фигурные скобки ставятся в любом случае.
Для else фигурные скобки ставятся случае если действий более одного.\\
\\\\
\begin{tabular}{ | l | }
\hline
\\
1 \textcolor{blue}{if} (x > y)\{\\
2 \qquad cout << "Первое число больше";\\
3 \} \textcolor{blue}{else} cout << "Второе число больше";
}\\
\\
\hline
\end{tabular}\\
\\\\
\begin{tabular}{ | l | }
\hline
\\
1 \textcolor{blue}{switch} (x)\{\\
2 \qquad \textcolor{blue}{case} 1:\\
3 \qquad\qquad ...;\\
4 \qquad\qquad continue;\\
5 \qquad \textcolor{blue}{case} 2:\\
6 \qquad\qquad ...;\\
7 \qquad\qquad break;\\
8 \} 
}\\
\\
\hline
\end{tabular}\\
\\\\
\end{lstlisting}

\subsection{Структуры}
\begin{lstlisting}
Cтруктуры, объявляются с помощью typedef. Сами названия пишутся с большой буквы.\\
\\\\
\begin{tabular}{ | l | }
\hline
\\
1 \textcolor{blue}{typedef struct} Node \{\\
2 \qquad\textcolor{blue}{int} vaule;\\
3 \qquad\textcolor{blue}{struct} Node *left;\\
4 \qquad\textcolor{blue}{struct} Node *right;\\
5 \} Node;
\\
\hline
\end{tabular}\\
\\\\
\end{lstlisting}

\subsection{Классы}
\begin{lstlisting}
 Отделяйте ваши объекты, делая все поля данных в вашем классе private. Всегда размещайте объявления классов и их частей в собственные файлы, ClassName.h.\\
 \\\\
\begin{tabular}{ | l | }
\hline
\\
1 \textcolor{blue}{class} MyClass\{\\
2 \qquad \textcolor{blue}{public}:\\
3 \qquad\qquad \textcolor{blue}{int} value;\\
4 \qquad \textcolor{blue}{private}:\\
5 \qquad\qquad \textcolor{blue}{int} getValue()\{\\
6 \qquad\qquad\qquad return value;\\
7 \qquad\qquad\}\\
8 \} 
}\\
\\
\hline
\end{tabular}\\
\\\\
\end{lstlisting}

\section{Вывод}
Я освоил LaTex и научился им пользоваться. Составил CodeStayle.

\newpage

\begin{thebibliography}{}
\bibitem{1}Code style как стандарт разработки. - URL:\\           [https://habr.com/ru/company/manychat/blog/468953/]
\bibitem{2}Code style как стандарт разработки. - URL:\\           [https://tproger.ru/translations/stanford-cpp-style-guide/]
\bibitem{3}Code style как стандарт разработки. - URL:\\           [https://habr.com/ru/company/ruvds/blog/574352/]
\end{thebibliography}

\end{document}
